The transactional memory is an augmented version of the LL/SC scratchpad memory.
The main idea is to be able to write only when all the read variables (that have
not been written to yet) remained untouched.

A transaction is started by reading from the transactional memory; the commit
process begins when the first write occurs and terminates when all the read
variables have been written to; the transaciton is then finished.
This saves us from introducing new instructions.

We achieve this by having a more complex state for the memory locations: they
can be \textit{written to after the start of a transaction},
\textit{written to before the start of a transaction} or
\textit{untouched since the start of a transaction}. Relevant aggregated
information for the bits is derived by a clever encoding for these states and
the use of OR gates for all the status bits relative to a single core.
The first state is encoded as \texttt{10}; the second as \texttt{00}; the third
as \texttt{01}. This way, by OR-ing all the most significant bits for a core
we know whether any memory block has been written to since the start of a transaction;
on the other hand, by OR-ing the least significant bits in the same way, we
derive when all the variables that have been read have been also written to.

Writes are only allowed if no location is in the \textit{written to after the start of a transaction}
state; if this is not the case, then a specific error is returned.
Furthermore, neither reads nor writes are allowed when a core is in the commit
process, but only for the locations that have not yet been commmitted, that is
still in the \textit{untouched since the start of a transaction} state. This
ensures that:

\begin{itemize}
    \item Only consistent data is read: in fact, even if locations are available
        for reading before the end of a commit, those are only the ones that have
        actually been written to.
    \item The conditions under which a transaction starts to commit stay unchanged
        throughout the commmit process: indeed, it the read variables can not
        be written, their state can not change and with it the precondition for
        committing.
    \item Commits can not interleave: there is no point in time at which locations
        
\end{itemize}

If a write operation is carried out, then the status bits for the location are
written as follows: for every core other than the one executing the command, if
the status bit itself is \textit{untouched since the start of a transaction},
then it is set to \textit{written to after the start of a transaction},
otherwise, it is left untouched; for the core itself, it is set to
\textit{written to before the start of a transaction}.

\textit{Atomicity}
  \begin{itemize}
    \item Commit
    \item Abort
  \end{itemize}
